\begin{tikzpicture}[
    declare function={
        normal(\x,\mu,\sigma) = 1/(\sigma*sqrt(2*pi))*exp(-(\x-\mu)^2/(2*\sigma^2));
    }
]
    \begin{axis}[
        axis lines = middle,
        ylabel = {$f(x)$},
        xlabel = {$x$},
        ymax = 0.45,
        ytick = \empty,        % Elimina las marcas del eje Y
        yticklabels = \empty,
        xmin = 0,
        xmax = 8,
        domain = 0:8,
        xtick = {1,2,3,4,5,6,7},
        xticklabels = {$-3\sigma$, $-2\sigma$, $-\sigma$, $\mu$, $\sigma$, $2\sigma$, $3\sigma$},
        xticklabel style={rotate=45, anchor=east, font=\footnotesize},
       % grid = major,
        %grid style = {dashed, gray!30}
    ]
    
    \def\muu{4}
    \def\sigmaa{1}
    
    % Gráfica de la distribución normal
    \addplot[color=mygreen, thick, samples=80, smooth] {normal(x,\muu,\sigmaa)};
    
    % Ciclo for para marcar los intervalos sigma (VERSIÓN CORREGIDA)
    \foreach \n in {-3,-2,-1,0,1,2,3} {
        \pgfmathsetmacro\xpos{\muu + \n*\sigmaa}
        \pgfmathsetmacro\ypos{normal(\xpos, \muu, \sigmaa)}
        
        % Línea vertical punteada
        \addplot[dashed, black] coordinates {(\xpos, 0)  (\xpos, \ypos)};
    }
    
    % Etiqueta para el valor máximo (OPCIONAL)
    % \node[above] at ({\muu}, {normal(\muu, \muu, \sigmaa)}) 
    %    {$\dfrac{1}{\sigma\sqrt{2\pi}}$};
    
    \end{axis}
\end{tikzpicture}
